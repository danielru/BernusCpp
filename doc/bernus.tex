\documentclass{scrartcl}

\usepackage{geometry}
\geometry{paper=a4paper,headheight=5mm, footskip=7.0mm, footnotesep=2mm, margin=20mm, tmargin=15.25mm, bmargin=15.0mm, includeheadfoot}

\usepackage{amsmath}
\usepackage{amsthm}
\usepackage{amssymb}
\newtheorem{theorem}{Theorem}
\newtheorem{lemma}{Lemma}
\newtheorem{prop}{Proposition}
\newtheorem{remark}{Remark}
\newcommand{\super}[1]{\textsuperscript{#1}}
\usepackage{enumitem}

\def\sodcurr {$i_{\rm Na}$}
\def\calcurr {$i_{\rm Ca}$}
\def\tocurr {$i_{\rm to}$}
\def\potcurr {$i_{\rm K}$}
\def\pincurr {$i_{\rm K1}$}
\def\cabacurr { $i_{\rm b, Ca}$ }
\def\nabacurr { $i_{\rm b, Na}$ }
\def\napopump { $i_{\rm Na, K}$ }
\def\nacapump{ $i_{\rm Na, Ca}$ }

\begin{document}

\title{Bernus model}
\author{Daniel Ruprecht}

\maketitle

\section{Bernus model}
From~\texttt{https://models.cellml.org/}. 

\subsection{Currents and pumps}
There are 9 currents, each employing different gating variables. 
These gating variables depend on the membrane potential $V$.
Below, we distinguish between so-called ''stationary'' gates which are given by algebraic equations and time-dependent gates, which are described by ODEs and therefore have to be included in the time-stepping procedure.
\begin{table}[th]
\centering
\begin{tabular}{cc}
\multicolumn{2}{c}{\bf Currents and pumps}\\
Sodium current & \sodcurr \\
Calcium current & \calcurr \\
Transient outward current & \tocurr \\
Delayed rectifier potassium current & \potcurr \\
Inward rectifier potassium current & \pincurr \\
Calcium background current & \cabacurr \\
Sodium background current & \nabacurr \\
Sodium potassium pump & \napopump \\
Sodium calcium pump & \nacapump
\end{tabular}
%
\caption{Currents and pumps in the Bernus model.}
\end{table}
Forcing term for monodomain equation reads
\begin{equation}
	C_{\rm m} I_{\rm ion}(V) = i_{\rm Na} + i_{\rm Ca} + i_{\rm to} + i_{\rm K} + i_{\rm K1} +  i_{\rm b, Ca} + i_{\rm b, Na}  + i_{\rm Na, K} + i_{\rm Na, Ca}
\end{equation}
The currents and pumps are defined as
\begin{enumerate}
\item {\bf Sodium current}
\begin{equation}
	i_{\rm Na} = g_{\rm Na} m^3 v^2 \left( V - E_{\rm Na} \right)
\end{equation}
where $E_{\rm Na}$ is the equilibrium current and $m$ and $v$ are gating variables.
\item {\bf Calcium current}
\begin{equation}
	i_{\rm Ca} = g_{\rm Ca} d_{\infty} f f_{\rm Ca} \left( V - E_{\rm Ca} \right)
\end{equation}
with $E_{\rm Ca}$ the equilibrium current, $f$ a time-dependent gate and  $d_{\infty}$ and $f_{\rm Ca}$ ''stationary'' gating variables, i.e. given by simple algebraic equations instead of ODEs.
\item {\bf Transient outward current.}
\begin{equation}
	i_{\rm to} = g_{\rm to} r_{\infty} to \left( V - E_{\rm to} \right)
\end{equation}
with equilibrium current $E_{\rm to}$, time-dependent gate $to$ and stationary gate $r_{\infty}$.
\item {\bf Delayed rectifier potassium current.}
\begin{equation}
	i_{\rm K} = g_{\rm K} X^2 \left( V - E_{\rm K} \right)
\end{equation}
with equilibrium current $E_{\rm K}$ and time-dependent gate $X$.
\item {\bf Inward rectifier potassium current.}
\begin{equation}
	i_{\rm K1} = g_{\rm K1} K_{1, \infty} \left( V - E_{\rm K} \right)
\end{equation}
with equilibrium current $E_{\rm K}$ and stationary gate $K_{1, \infty}$.
\item {\bf Calcium background current.}
\begin{equation}
	i_{\rm b, Ca} = g_{\rm b, Ca} \left( V - E_{\rm Ca} \right)
\end{equation}
with equilibrium current $E_{\rm Ca}$ and no gating variables.
\item {\bf Sodium background current.}
\begin{equation}
	i_{\rm b, Na} = g_{\rm b, Na} \left( V - E_{\rm Na} \right)
\end{equation}
with equilibrium current $E_{\rm Na}$ and no gating variables.
\item {\bf Sodium potassium pump.}
\begin{equation}
	i_{\rm Na, K} = g_{\rm Na, K} f_{\rm Na, K} f_{\rm Na, K, a}
\end{equation}
with stationary gating variables $f_{\rm Na, K}$, $f_{\rm Na, K, a}$.
\item {\bf Sodium calcium pump.}
\begin{equation}
	i_{\rm Na, Ca} = g_{\rm Na, Ca} f_{\rm Na, Ca}
\end{equation}
with a stationary gate variable $f_{\rm Na, Ca}$.
\end{enumerate}
%
%
\subsection{Gating variables}
\begin{table}[ht]
\begin{tabular}{cc}
\multicolumn{2}{c}{\bf ODE-based Gate variables}\\
bla
\end{tabular}\\
%
\begin{tabular}{cc}
\multicolumn{2}{c}{\bf Diagnostic gate variables}\\
blub
\end{tabular}
\end{table}
\end{document}